\section{Related Work}


% TODO Note: I plaigarized the wording on these definitions from wikipedia at this point, should pull from original Osborne at some point
Brainstorming was originally proposed in XXXX by Osborne \cite{osborn_applied_1957}. Brainstorming makes several recommendations for idea generation in groups, the core of which are a set of four rules: \emph{focus on quantity}, \emph{withhold criticism}, \emph{welcome unusual ideas}, and \emph{combine and improve ideas}.

Since then, brainstorming as proposed by Osborne has been extensively evaluated. Very quickly, Taylor et. al established that \emph{nominal groups}, or individuals brainstorming in isolated whose ideas are later combined, performed better than real groups in terms of number of ideas generated \cite{taylor_does_1958}. Furthermore, Diehl and Stroebe found that groups which generated good ideas also generated many ideas \cite{diehl_productivity_1987}, a finding that has been verified many times since (Another citation here would be nice). In Bouchard and Hare \cite{bouchard_jr_size_1970}, nominal brainstorming groups were able to generate quantities of ideas linear in the number of members in the group.

Electronic brainstorming is a variant in which individuals brainstorm at separate computer terminals, with the exchange of ideas performed over the network. Participants can enter ideas while simultaneously viewing the ideas of all other ideas, anonymously presented on-screen. Gallupe \cite{gallupe_electronic_1992} found that electronic brainstorming reduced the inhibiting social effects of \emph{production blocking} (the inability for multiple people to speak at once), and \emph{evaluation apprehension} (the fear of judgment) in comparison to in-person groups. However, Pinsonneault et al identify four productivity impediments introduced by group electronic brainstorming, and show that there is little evidence to support the proposition that electronic brainstorming out-performs nominal brainstorming \cite{pinsonneault_electronic_1999}.

As crowd marketplaces do not tend to include built-in support for real-time communication between workers, nominal brainstorming groups provide the most reasonable assumption for providing a baseline assessment of performance. It is fortuitous, then, that prior work suggests that nominal groups are likely equivalent at worst to groups with the ability to communicate.

This choice is supported by other users of crowd marketplaces for idea generation. Yu and Nickerson use many individuals in an evolutionary algorithm to generate creative chair designs \cite{yu_cooks_2011}. This technique is compared to single participants generating single ideas, but no baseline is established for idea generators individually exhausting a design space.

Something about another example. Maybe the Little iterative vs parallel paper? It includes creative tasks.


% TODO: part of this is Pao-exact, assuming it's cool because she's going to be an author on the paper
While there are no models of nominal brainstorming in the context of a crowd marketplace, several researchers have modelled the idea generation process for individuals and groups. We emphasize the former for its applicability to nominal groups. Nijstad and Stroebe define the \emph{search for ideas in associative memory} (SIAM) model \cite{nijstad_how_2006}. SIAM assumes two stages of idea generation: knowledge activation based on search of working memory, and idea production within a single activated image. This model verifies three predictions for individual brainstorming behaviour:

\begin{enumerate}
\item an idea is more likely to be followed by an idea in the same category than would be expected by chance
\item new ideas in the same category as the previous idea are generated faster than ideas between categories
\item increased clustering (generation of ideas within-category) is positively correlated with overall high production
\end{enumerate}

Should we have another model here? I haven't read any yet.

In summary, while brainstorming has been examined in the traditional and electronic settings, its application in crowd marketplaces has received limited attention. 