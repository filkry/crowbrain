\section{Related Work}
In this section, we review related work in the areas of brainstorming and crowdsourcing.

% TODO Note: I plaigarized the wording on these definitions from wikipedia at this point, should pull from original Osborne at some point
In the 1950s, Osborne formalized the brainstorming process, providing a set of recommendations for idea generation by groups \cite{osborn_applied_1957}: \emph{focus on quantity}, \emph{withhold criticism}, \emph{welcome unusual ideas}, and \emph{combine and improve ideas}. These recommendations form the basis for the instructions we provide to workers in our experiments.

Following Osborne's work, Taylor et al.\ established that brainstorming by \emph{nominal groups}, or individuals working in isolation from one another, yields better results than collocated groups in terms of number of ideas generated \cite{taylor_does_1958}. These effects can be partially explained by the absence of counterproductive social forces, such as the fear of judgment. Bouchard and Hare \cite{bouchard_jr_size_1970}, later found that nominal brainstorming groups are able to generate quantities of ideas linear in the number of members in the group.

Electronic brainstorming is a variant in which individuals brainstorm at separate computer terminals, with the exchange of ideas performed over the network \cite{gallupe_electronic_1992}. Participants can enter ideas while simultaneously viewing the ideas of all other ideas, which are anonymously presented on-screen. As with nominal brainstorming, Gallupe \cite{gallupe_electronic_1992} observed that electronic brainstorming can reduce counterproductive social effects, such as production blocking (the inability for multiple people to speak at once) and fear of judgment, compared to physically collocated groups. However, Pinsonneault et al.\ \cite{pinsonneault_electronic_1999} identify productivity impediments introduced by group electronic brainstorming, such as being distracted by other ideas appearing, or individuals limiting their idea generation to avoid replicating the ideas of others. Given these drawbacks, their results suggest there is little evidence to support the proposition that electronic brainstorming outperforms nominal brainstorming.

Finally, Diehl and Stroebe found that groups that generate good ideas also generate \emph{many} ideas \cite{diehl_productivity_1987}, a finding that has been verified many times \cite{briggs1997quality, parnes1959effects, parnes_effects_1961, shah2003metrics, cross1996creativity}.

In terms of individual brainstorming processes, Nijstad and Stroebe provide a model for individual idea generation dubbed SIAM, or the search for ideas in associative memory. \cite{nijstad_how_2006}. SIAM defines two stages of idea generation: the generation of ideas based working memory, and idea production within a single activated ``image'' (where an image can be thought of as a category of ideas). This model predicts the following:

\begin{enumerate}
\item an idea is more likely to be followed by an idea in the same category than would be expected by chance
\item new ideas in the same category as the previous idea are generated faster than ideas between categories
\item increased clustering (generation of ideas within-category) is positively correlated with overall high production
\end{enumerate}

Later, we will present evidence in support of this model from the data we collected.

In the realm of crowd-based brainstorming, there are a number of ways this process could be realized. The most basic configuration mirrors that of nominal brainstorming, where individuals brainstorm ideas in isolation of one another, and the resulting ideas are passed on to the requestor (the individual posting the brainstorming task). More sophisticated brainstorming environments are possible, such as those that provide awareness of ideas that others have generated, or those that rely on the crowd to organize or rank the ideas produced by others. Examples for these environments are platforms for generating social innovation ideas or product concepts like OpenIDEO, Quirky and Threadless that employ structured process where participants take on different responsibility from generating ideas to evaluating others' ideas. In this paper, we focus on the most basic setup, where individuals brainstorm in isolation of one another.

While we are unaware of specific studies of brainstorming processes in crowd marketplaces, there is a range of related research examining more creative processes in this environment \cite{lewis2011affective, kittur2011crowdforge, Zhang:2012:HCT:2207676.2207708}. For example, Yu and Nickerson created a system in which individuals participate in an evolutionary algorithm to generate creative chair designs \cite{yu_cooks_2011}. Compared to ideas produced by single individuals, they found the ideas produced as a result of their genetic algorithm were more creative. However, individuals produced only one idea each, compared to the many ideas typically produced by individuals in a brainstorming session. Thus, while their approach demonstrates the creative potential of the crowd, their results say little about brainstorming processes by crowds. 

Little et el. compared the iterative and parallel human computation processes on a brainstorming task for names for fabricated companies \cite{little2010exploring}. In the iterative condition, participants saw all names suggested by previous participants. Participants in the parallel condition worked in isolation without seeing other participants' ideas. Subjective ratings of the quality of ideas reveal that while the ideas from the iterative condition were rated higher on average, the best ideas from the parallel condition have higher scores than the best ideas from the iterative conditions. The parallel process is more likely to generate names with high ratings, advocating the case for nominal brainstorming over traditional electronic brainstorming. However, the study did not mention categorization of generated ideas, an important measure on flexibility of idea generation \cite{lewis2011affective, nijstad_how_2006, finke1992creative, shah2003metrics}. Each participant also generated only five ideas, leaving the number of ideas requested per participant an open dimension to be explored.

The nature of microtask marketplace also calls for an alternative structure for brainstorming tasks in this setting. Researchers have found that 30\% or more submission on Mechanical Turk may be low quality \cite{kittur2008crowdsourcing}. For example, submissions to a brainstorming task for a company name are sometimes grammartically awkward or offensive \cite{little2010exploring}. These low quality responses were not found in walk-in brainstorming studies. Bernstein et al. identified high variance of effort Amazon Mechanical Turk workers put on task \cite{soylent}. They characterized two worker personas at opposite ends of the effort spectrum: the {\em Eager Beaver\/} who goes beyond task requirement and the {\em Lazy Turker\/} who does as little work as necessary to get paid. A brainstorming task that asks participants to generate as many ideas as they can under a set time limit might not be practical. A {\em Lazy Turker\/} can simply fill in a few ideas and wait for the time to run out instead of exerting full effort. The motivational structure of microstask marketplace does not map well with most previous work on brainstorming where the sessions are time-limited. 


%TODO: More crowdsourcing related work here. Maybe the Little iterative vs parallel paper? It includes creative tasks. Perhaps describe some of the unique characteristics of microtask marketplaces, like people's desire to perform a task and move on...

In summary, while brainstorming has been examined in the traditional and electronic settings, its application in crowd marketplaces has received limited attention. 