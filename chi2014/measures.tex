\section{Measures}

\subsubsection{Originality}

As our measure of originality, we use \emph{o-score}, introduced by Jansson and Smith \cite{jansson_design_1991}. An idea's o-score is $1 - p(idea)$, where $p(idea) = (number of instances of that idea)/(number of instances total)$. The o-score for category trees is similarly calculated. We will occasionally refer to the \emph{category o-score} of an idea. This simply refers to the o-score of the category tree to which that idea belongs.

\subsubsection{Uniqueness}

To examine the quantity of idea produced, we must first have some method of comparing two ideas to see if they are unique. We use three such comparisons. The first, \emph{idea equality}, equates two instances if they belong to the same idea node. The second, \emph{category equality}, equates two instances if their idea nodes are in the same tree. The former requires exact equality in meaning, while the latter requires only that the same strategy is employed.

The final measure, \emph{look-back equality}, is non-symmetrical and has meaning only in the context of a single brainstorming run. An instance \emph{a} is look-back equal to an \emph{earlier} instance \emph{b} if \emph{b} is in either \emph{a}'s idea node or any of \emph{a}'s parents, siblings, or children. This definition specifically identifies when a later instance in a run can be considered to have been influenced by earlier instance.