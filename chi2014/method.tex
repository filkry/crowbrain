We began the process of soliciting responses to brainstorming tasks in a crowd marketplace. Throughout the experimental process, particular aspects of the experimental design remained fixed.

The independent variables of interest were \emph{brainstorming problem} and \emph{number of ideas requested}.

Participants were recruited from Amazon's \emph{Mechanical Turk} (CITE), and were restricted to residents of the United States for a baseline expectation of english language comprehension and cultural familiarity. Mechanical Turk is an online marketplace in which members receive financial reward for completing \emph{Human Intelligence Tasks}, or HITs.

HITs were placed on the marketplace for each \emph{number of ideas requested} condition, with proportional rewards. Participants could accept at most one HIT in each of these conditions.

Upon accepting a HIT, participants are randomly assigned into a \emph{brainstorming problem} condition. We ensured that participants completing multiple HITs are not exposed to the same brainstorming task twice.

\subsubsection{Design Concerns}
Workers choose which HITs to complete, thus self-selection bias is a reasonable concern. This bias is also present in real-world HIT choice behaviour. 

Upon accepting a HIT, participants were asked to give consent and informed that they could leave the study at any point without financial consequences. In regular online task markplaces, common practice would restrict workers from submitting a HIT as complete unless all responses were given. This is a threat to external validity.

\subsection{Task}

The brainstorming task is a form in a standard web browser. Participants are presented with a brief overview of the tenets of brainstorming, are presented a problem, and must enter some number of ideas to resolve the problem within an 18 hour period.

At the beginning of the form is a brief introduction to brainstorming, with a paraphrase of Osborne's four rules of brainstorming (CITE). These rules were manipulated to make sense within the constraints nominal brainstorming over the web medium. The rules, as displayed to the participants:

\begin{enumerate}
\item There are no bad ideas. Don't criticise your choices.
\item Wild ideas and building off of old ideas are okay.
\item Quantity of ideas is prioritized.
\item Combinations of ideas count as new ideas.
\end{enumerate}

The brainstorming task is below this, followed by a series of text entry inputs numbered through to the total number of ideas requested. Figure X (FIG) is an example of a typical task. We place a larger free text area at the bottom of the list where participants could enter any additional ideas.

\subsection{Pilots and Question Selection}

With this basic design in hand, we ran several pilots and experiments. In early pilots, we used classic problems from psychology literature on brainstorming, including the "thumb problem" and "broom problem" (CITE). Early results from brainstormers were unsatisfying and wildly divergent. We identified three key traits problems needed in order to evaluate them for creativity:

\begin{enumerate}
\item They must require creativity to resolve. Obvious answers do not satisfy the problem.
\item They must be problems that participants on Mechanical Turk would have the expertise to solve.
\item They must have an associated success metric.
\end{enumerate}

The primary researcher and two additional researchers familiar with crowd marketplace brainstorming tasks brainstormed a large variety of potential problems. These problems were iterativelly culled and refined over a series of further pilots until the above goals were felt to be achieved. This resulted in the following four questions, as presented to the participants:

\begin{enumerate}
\item \textbf{Charity}

The Electronic Frontier Foundation (EFF) is a nonprofit whose goal is to protect individual rights with respect to digital and online technologies. For example, the EFF has initiated a lawsuit against the US government to limit the degree to which the US surveils its citizens via secret NSA programs. If you are unfamiliar with the EFF and its goals, read about it on its website (https://www.eff.org) or via other online sources (such as Wikipedia).

Brainstorm N \emph{new} ways the EFF can raise funds and simultaneously increase awareness. Your ideas \emph{must be different from their current methods}, which include donation pages, merchandise, web badges and banners, affiliate programs with Amazon and eBay, and donating things such as airmiles, cars, or stocks. See the full list of their current methods here: https://www.eff.org/helpout. Be as specific as possible in your responses."

\item \textbf{Mechanical Turk}

"Mechanical Turk currently lacks a dedicated mobile app for performing HITs on smartphones (iPhone, Androids, etc.) or tablets (e.g., the iPad).

Brainstorm N features for a mobile app to Mechanical Turk that would improve the worker's experience when performing HITs on mobile devices. Be as specific as possible in your responses.

\item \textbf{MP3}

Many people have old iPods or MP3 players that they no longer use. Please brainstorm N uses for old iPods/MP3 players. Assume that the devices' batteries no longer work, though they can be powered via external power sources. Also be aware that devices may \emph{not} have displays. Be as specific as possible in your descriptions.

\item \textbf{Forgot Name}

Imagine you are in a social setting and you have forgotten the name of somebody you know. Brainstorm N ways you could learn their name without directly asking them. Be as specific as possible in your descriptions.

\end{enumerate}

Here do I talk about the unlimited condition?

\subsection{Experiment}

With the above problem set and the previously outlined task design (SEC), we collected responses. We chose 6 \emph{number of ideas requested} conditions that covered the spectrum (and beyond) of quantity of creativity requested from a single participant. Those conditions were: 5 ideas (with a corresponding reward of \$0.18 USD), 10 ideas (\$0.35), 20 (\$0.70), 50 (\$1.75), 75 (\$2.65), and 100 (\$3.50). To reiterate, participants chose a HIT to generate some number of ideas, and then were randomly assigned a problem.

\subsubsection{Data Collection}

Does it make sense to have a section here on what data we collected?

\subsection{Coding}

One researcher coded the data for category in a hierarchical scheme. For each question, ideas recieved from the participants were organized in a tree structure. A single node in the tree represents a particular solution to the problem.

To formalize the clustering method, we introduce some key terms:

\subsubsection{Similarity}
Similarity is a subjective measure of the equality of two nodes a, b on the basis of their key problem solution. a and b have high similarity if they solve the problem in the same way, and low similarity if their solutions have no common themes.

\subsubsection{Coverage}
A node a has high coverage of a node b if the problem solution in a is equal to or an abstraction of the problem solution in b. Coverage is not commutative.

\subsubsection{Symmetrical Coverage}
Two nodes a, b have symmetrical coverage if coverage(a, b) = coverage(b, a).

In the resulting tree, all parent nodes have high coverage over their child nodes, but all child nodes have low coverage over their parent nodes. All nodes have high affinity with their parent and their children. The clustering algorithm is outlined in figure (FIG).

\begin{figure*}[h]
\begin{verbatim}
for each idea:
  idea_node = new cluster with idea as member
  current_node = root
  do:
    best_match = max_affinity(idea_node, current_node.children)

    if best_match.affinity is low or current_node has no children:
      insert idea_node under current_node
      exit do
    else:
      if symmetrical_coverage(idea_node, best_match) and high:
        merge idea_node, best_match
        exit do
      else symmetrical_coverage(idea_node, best_match) and low:
        new_parent = new child node of current_node, summarizing cause of affinity
        insert best_match, idea_node under new_parent
        exit do
      else if coverage(idea_node, best_match) > coverage(best_match, idea_node):
        replace best_match with idea_node in tree
        current_node = idea_node
        idea_node = best_match
      else:
        current_node = best_match
\end{verbatim}
\caption{Manual clustering algorithm}
\label{fig:cluseringalg}
\end{figure*}