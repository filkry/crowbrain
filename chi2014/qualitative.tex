\section{Qualitative findings}

In addition to modelling data and confirming our hypotheses, we performed a qualitative analysis. The primary goal was to identify trends  that were repeated between participants. These trends can then be used to inform and justify interventions to change the types of ideas produced.

We selected 30 brainstorming runs (5 randomly from each condition). The primary author applied an open coding strategy, flagging ideas within runs that shared common characteristics, such as length, wording, or core problem solution.

In particular, we identify classes of ideas within runs, often temporally close, that hold common characteristics. For each class that was observed across multiple participants, we infer the existence of a \emph{strategy}: a plan or rule that aids in generating ideas. We describe the observed strategies in detail below.
Furthermore, we use the presence or lack of presence of strategies to define \emph{personas}.

\subsection{Strategies}

A \emph{strategy} is a plan, rule, or idea that is used by a participant to generate brainstorming responses.
Without debriefing participants, we cannot infer the goal of a strategy, but define it in terms of its observable effects.
These effects are repeated, cross-participant presence of classes of similar ideas.
Not all participants utilized all strategies in their brainstorming runs.

We identified X \emph{orthogonal} strategy categories - each category of strategies can be employed by a participant with or without any of the other categories.
The first, \emph{problem scoping}, is a process in which a participant chooses a limited implication or component of the problem and provides multiple specific solutions.
The second category expands on the \emph{riffing} phenomenon introduced earlier, identifying X different types of riffing exployed.
Finally, \emph{language use} strategies... TODO
%Finally, the \emph{humour} strategy emphasizes generating ideas that are amusing rather than practical.

\subsubsection{Problem scoping}

To complete the brainstorming task, participants had to generate some number of uses for a partially broken MP3 player. Under the problem scoping strategy, this problem can reduced in scope to one of many sub-problems: brainstorm uses for \emph{an audio output device}; brainstorm uses for \emph{an electronic screen}; brainstorm uses for \emph{a hard drive}; and so on. The participant chooses one sub-problem at a time and generates solutions, each of which is also a solution to the original problem. For example, the following are generated by a participant focussing on the audio output capabilities of an MP3 player:

\begin{itemize}
    \item have them as a resource on public transportation -- people must supply their own headphones
    \item use them in museums to give information on various installations
    \item have cities install them in tourist areas, so people can listen about where they are
\end{itemize}

The previous examples of problem scoping are examples of emphasizing specific details of the problem. Problem scoping can also entail neglecting details. For example, the following ideas from a participant apply to almost any physical object, not just the MP3 player described:

\begin{itemize}
    \item doorstop
    \item use in abstract art
    \item recycle
\end{itemize}

We refer to the former, detail-oriented kind of problem scoping as \emph{scoping inward} and the alternative as \emph{scoping outward}. Problem scoping is employed by all participants, but the type of scoping (inwards vs outward) is a useful discriminant.

\subsubsection{Riffing}

The second category of strategies is \emph{riffing}.
We found riffing manifested itself many ways, and identified four distinct riffing strategies used by participants: generalization riffing, repeat riffing, hold riffing, and continuation riffing. We describe each of these in detail, and then touch on three common phenomena found under riffing strategies.

\emph{Generalization riffing} occurs when a participant generates two or more ideas and one is a generalization of the other. For example, two consecutive ideas given by a participant:

\begin{itemize}
    \item brick
    \item building material
\end{itemize}

Another type of riffing is \emph{repeat riffing}. Under this strategy, a participant produces multiple reponses with only slight differences between them. For example:

\begin{itemize}
    \item recycle to make better MP3/ipod players
    \item use the circuits to for the ipod/MP3 to power other things that are more recent
\end{itemize}

In this case, the participant is providing the same idea (namely, creating new electronic devices from the parts) twice.

\emph{Hold riffing} is a strategy in which participants hold at least one element of a previous response constant when generating a new response, such that a new idea is formed. This can be thought of as analogous to the game of \emph{draw poker}, in which players hold some cards while receiving new ones, with the intent of building a winning hand complementing the held cards.
For example:

\begin{itemize}
    \item Jukebox music selector in bars
    \item Commercials in bar bathrooms (hold the bar element)
    \item Tapper handles for beer (hold the alcohol element)
\end{itemize}

Hold riffing can also involve holding language terms constant even when the meaning of the term changes between responses, as in free association. For example:

\begin{itemize}
    \item beer coaster
    \item drink coaster (generalization riffing)
    \item roller coaster (hold riffing)
\end{itemize}

\emph{Continuation riffing} is the strategy of creating another idea that assumes the previous idea has already been implemented. The new ideas relies on the implications or consequences of the previous. For example:

\begin{itemize}
    \item I suppose you could just grind them down into a sand
    \item You could take the sand... and put it in an hourglass
\end{itemize}


Within and across these strategies, we identified two sub-strategies or tweaked behaviours.
Some participants \emph{pair} their riffs. These participants  often riff on old ideas but never produce more than two versions of each idea. For example, paired hold riffing from a participant:

\begin{itemize}
\item TODO
\item TODO
\end{itemize}

Finally, \emph{reaching back} is a variant of riffing employed by almost all participants who provided more than 10 responses. Reaching back entails riffing on an idea that occured further back in the run than the previous response. The presence of reaching back speaks to the exhaustion of the participant's creativity, as they produce new variants of old ideas rather than new ideas altogether.

Almost all participants employ some form of riffing. This is not surprising, as it is ingrained in the rules of brainstorming given at the beginning of the task: "combinations of ideas count as new ideas". 

\subsubsection{Language use}

Participants varied in the language they used to express their ideas. Three strategies were observed: language repitition, brevity, and ambiguity.

Some participants re-use sentence structure or phrasings accross multiple responses. We call this strategy \emph{language repitition}. For example:

\begin{itemize}
    \item We could use the hard-drives inside for different electronics.
    \item We could use them in place of rocks (to throw at things, to use in pavement.)
    \item We could use them to back up our music collections on our current devices.
\end{itemize}

In this example, while language is re-used, each instance is a new idea.
However, language repitition is especially common in repeat riffing, as this example shows:

\begin{itemize}
    \item wrap yarn around it
    \item wrap twine around it
    \item wrap string around it
\end{itemize}

\emph{Brevity} is the simple strategy of producing short responses. Answers under this strategy are not neccessarily less specific than longer answers. For example "textbook audio reader" is a concise description of a very specific use for an MP3 player.

\subsubsection{Partial solutions}

Under this category of strategies, responses provide some elements of a solution to the problem, but further idea generation is required to implement the solution. Some responses provide a reduced scope, but do not explore within that scope. Others would provide a goal for a solution to obtain but not provide a plan for achieving that goal. Finally, some responses passed the problem to a third party.

Above, we described the problem scoping strategy. Some participants performed \emph{problem scoping without solution}, a variant of this strategy in which problem scoping is the only component of the response.
For example, one participant's response to the MP3 player problem was to "Remove glass screen to make something". In this case, the participant has provided a scoping of the problem (focus on the glass component) but has not actually provided a solution to the scoped problem.

When a solution is provided, it is still occasionally lacking in sufficient information to act upon. We call this strategy \emph{goal without plan}. For example, a common response to the problem was "use as art". In this case, an end goal has been established, but any implementation of the goal requires further idea generation. The inverse of this strategy is \emph{plan without goal}, as in the response "change the settings". In this case, one could certainly perform the operation on the MP3 player, but it doesn't provide any obvious benefit.

In many cases, responses that do not have an explicit goal or plan nonetheless imply one. For example, "throw it in the ocean" is an action performed with the MP3 player with no stated goal, but the implied goal is to get rid of the broken device. In these cases, as judged by the coder, reponses were not considered part of a partial solution strategy.

Finally, responses in the \emph{pass the problem} strategy simply relocate the need for idea generation to a third party. Common responses under this strategy are "give as gift" or "donate to cancer research". These solve the original problem, but simultaneously re-open the problem under slightly changed conditions. In this case, some cancer researchers must determine what to do with broken MP3 players.

\subsection{Personas}

A persona can be thought of as a toolbox of strategies that we see repeated across participants. Participants of a given persona have a subset of strategies that are used often, and a subset that are rarely used. In a process analogous to affinity diagramming, brainstorming runs were clustered based on similarity of strategy use. This resulted in six clusters of repeated behaviour, each of which is a basis of a persona. These personas are summarized in Table TAB.

The \emph{high concept} persona consists of participants who view the MP3 player as an abstract object and provide many distinct, if often vague, responses. Commonly employed strategies are scoping outward, hold riffing, and partial responses.

\emph{Outside the box} personas are similar to high concept, except that their responses are more well-defined. Common strategies are scoping outward, hold riffing and brevity. Partial responses are rare.

and so on...

\subsubsection{Outliers}

In addition to the observed personas, there were several outliers. One participant employed no strategies to a signficant degree (in the case of problem scoping, they scoped inward and outward with similar frequency). A second outlier responded with questions, such as "What can the average person do to better the environment?".

\begin{table*}
    \begin{tabular}{|l|p{0.35\linewidth}|l|}
        \hline
        \textbf{Persona} & \textbf{Common strategies} & \textbf{Rare strategies} \\
        \hline
        High concept & scoping outward, hold riffing, partial responses & brevity, reach-backs \\
        \hline
        Outside the box & scoping outward, hold riffing, brevity & partial responses, reach-backs \\
        \hline
        broad maximizers & repeat riffing, brevity, \newline
        partial responses  & reach-backs \\ 
        \hline
        broad explorers & hold riffing, partial responses & \\
        \hline
        kitchen sink & scoping outward, repeat riffing, brevity, \newline
        partial responses, reach-backs & \\ 
        \hline
        detail explorers & scoping inward, hold riffing & brevity, partial responses, reach-backs\\
        \hline
    \end{tabular}
    \caption{Strategies used by personas}
\end{table*}
