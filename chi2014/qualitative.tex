\section{Qualitative findings}

In addition to our quantitative analysis of the rate of idea generation, we also performed a qualitative analysis.
We selected 30 brainstorming runs (5 from each condition). The primary author applied open coding to identify characteristics of responses and runs that are independent of the uniqueness and originality measures observed in the quantitative section of this paper.
The characteristics primarily took the form of \emph{strategies} employed to different degrees by different participants.
We constructed \emph{personas} around the use of these strategies.

\subsection{Strategies}

A \emph{strategy} is an identifiable mechanism for idea generation that results in several ideas in a brainstorming run. A key component of a strategy is that it be present in some runs but not in others.

One strategy employed was \emph{feature exploration}. An MP3 player may have many features that distinguish it from any other electronic device of the same physical proportions: audio output (and potentially input), data storage, a screen, general purpose computing abilities (apps), connectivity (WiFi, USB), a camera, a unique appearance, GPS.
Many participants treated the MP3 player as a pocket-sized brick when generating ideas.
In contrast, the feature use strategy led to responses such as:

\begin{itemize}
    \item use them in museums to give information on various installations (audio output)
    \item use in a scavenger hunt, device must be plugged in at another location to see clue
\end{itemize}

Runs utilizing the feature exploration strategy produce responses that seem intuitively better than those that do not. It is satisfying to have a use for a device that "fits" the device as closely as possible, so that none of the functionality is "wasted". Despite this, the proportion of runs utilizing feature exploration seemed roughly constant across all other strategies, and the strategy does not form a basis for discrimination between any personas.

The second strategy identified is \emph{riffing}. While riffing was examined earlier in relation to time spent on idea generation, we found in our qualitative analysis that it was manifested in many ways. \emph{Specificity riffing} occurs when a participant generates two or more ideas and one is a generalization of the other. For example, two consecutive ideas given by a participant:

\begin{itemize}
    \item brick
    \item building material
\end{itemize}

Another, more common type of riffing is \emph{exhaustive mutation}. Under this strategy, a participant repeats a response nearly verbatim many times, each time replacing a single word or concept in such a way that no new ideas are formed. For example:

\begin{itemize}
    \item wrap yarn around it
    \item wrap twine around it
    \item wrap string around it
\end{itemize}

\emph{Exploratory riffing} is a strategy in which participants hold at least one element of a previous response constant when generating a new response, such that a new idea is formed. This can be chained many times in a kind of idea "telephone game" such that the final ideas no longer resemble the original source. For example:

\begin{itemize}
    \item Jukebox music selector in bars
    \item Commercials in bar bathrooms (retains bar element)
    \item Tapper handles for beer (retain alcohol element)
\end{itemize}

Finally, \emph{Follow-up riffing} occurs when a participants response assumes that the solution in one of their previous responses has already been implemented. For example:

\begin{itemize}
    \item I suppose you could just grind them down into a sand
    \item You could take the sand... and put it in an hourglass
\end{itemize}

Follow-up riffing, while interesting, is incredibly rare in the data and we will not discuss it further.

There are two special variants of riffing that can occur across all riffing strategies. Some participants \emph{pair} their riffs, such that they often riff on old ideas but never produce more than two versions of each idea.

Another special case is is \emph{linguistic riffing}. Some participants seem to use common sentence structure as queues to help them generate ideas. For example:

\begin{itemize}
    \item Make sunglasses out of them
    \item Make shoes out of them
    \item Make a handbag out of them
\end{itemize}

Note that linguistic riffing and exhaustive mutation have significant overlap. A participant that is employing exhaustive mutation is always employing linguistic riffing, but not vice-versa.

Finally, \emph{reaching back} is a variant of riffing employed by almost all participants who provided more than 10 responses. Reaching back entails riffing on an idea that occured further back in the run than the previous response. The presence of reaching back speaks to the exaustion of the participant's creativity, as they produce new variants of old ideas rather than new ideas altogether.

Almost all participants employ some form of riffing. This is not surprising, as it is ingrained in the rules of brainstorming given at the beginning of the task: "combinations of ideas count as new ideas". However, the different sub-strategies of riffing have very different impacts on the qualitative assessment of a run. Exhaustive mutation is often unsatisfying, as it feels as though the participant is trying to wring every scrap of task completion out of a single idea. On the other hand, exploratory riffing can produce answers that are incredibly unexpected but still of high quality. Riffing strategies paint a significant picture of the persona of the participant.

Two time-saving strategies were also identified. The first, perhaps overly obvious strategy, is \emph{brevity}. This is simply the act of keeping responses as concise as possible, presumably in the interest of finishing quickly. The second time-saving strategy was \emph{cop-outs}. Cop-outs occur when participants choose to resort to vague or iunrelated responses that do not actually provide any solution to the problem. For example, one participant's response to the MP3 player problem was to "Remove glass screen to make something". Cop-out responses may provide one or more steps towards a solution, but no solution.

\subsection{Personas}

We use personas to capture collections of strategy use we see repeated across participants. While some strategies proved discriminatory, others were used more or less equally by all personas. Drill-down riffing, pairing and language riffing, while interesting strategies, are either minimal in the makeup of a run or have no impact of the subjective quality of the resulting ideas.

Six personas were identified: quitters, lazy creatives, diligent struggles, lazy strugglers, hard workers, and over-deliverers.

\emph{Quitters} are the easiest to summarize. These participants do not provide all the requested responses, and often resort to cop-outs or exhaustive mutation in the ideas that they do provide.

\emph{Lazy creatives} provide primarily uncommon responses, but do not put any particular effort into the responses. They employ the brevity, reach back and exhaustive mutation strategies to a great extent. These participants seem like prime targets for future interventions, as their ideas are of interest but they focus their strategies towards rapid completion. Lazy creatives employ minimal feature utilization.

\emph{Diligent strugglers} are the opposite of lazy creatives. These participants do not employ the brevity or exhaustive mutation strategies, but nonetheless produce primarily common, unoriginal ideas. These participants also seem like potential targets for interventions, as they apply themselves to the task but may need external assistance to move beyond their comfort zone. Strategies employed include exploratory riffing and feature utilization.

\emph{Lazy strugglers} are simply those who produce no creative ideas and are predisposed towards unproductive strategies (primarily exhaustive mutation and brevity).

\emph{Hard workers} are notable for their minimal use of riffing strategies. The riffing strategies they do use tend to exclude exhaustive mutation. While brevity is occasionally employed for short periods, it is generally only where a short description is sufficient to communicate an idea.

\emph{Over-deliverers} are notable for their lack of strategy in general. Their ideas are verbose, distinct and feature extremely minimal riffing. 
