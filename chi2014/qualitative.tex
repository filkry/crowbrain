\section{Qualitative findings}

In addition to modelling data and confirming our hypotheses, we performed a qualitative analysis. The primary goal was to identify trends  that were repeated between participants. These trends can then be used to inform and justify interventions to change the types of ideas produced.

We selected 30 brainstorming runs (5 randomly from each condition). The primary author applied an open coding strategy, flagging ideas within runs that shared common characteristics, such as length, wording, or core problem solution.

In particular, we identify classes of ideas within runs, often temporally close, that hold common characteristics. For each class that was observed across multiple participants, we infer the existence of a \emph{strategy}: a plan or rule that aids in generating ideas. We describe the observed strategies in detail below.
Furthermore, we use the presence or lack of presence of strategies to define \emph{personas}.

\subsection{Strategies}

A \emph{strategy} is a plan, rule, or idea that is used by a participant to generate brainstorming responses.
Without debriefing participants, we cannot infer the goal of a strategy, but define it in terms of its observable effects.
These effects are repeated, cross-participant presence of classes of similar ideas.
Not all participants utilized all strategies in their brainstorming runs.

We identified X \emph{orthogonal} strategy categories - each category of strategies can be employed by a participant with or without any of the other categories.
The first, \emph{problem scoping}, is a process in which a participant chooses a limited implication or component of the problem and provides multiple specific solutions.
The second category expands on the \emph{riffing} phenomenon introduced earlier, identifying X different types of riffing exployed.
Finally, \emph{time-saving} strategies increase the rate of idea generation.
%Finally, the \emph{humour} strategy emphasizes generating ideas that are amusing rather than practical.

\subsubsection{Problem scoping}

To complete the brainstorming task, participants had to generate some number of uses for a partially broken MP3 player. Under the problem scoping strategy, this problem can reduced in scope to one of many sub-problems: brainstorm uses for \emph{an audio output device}; brainstorm uses for \emph{an electronic screen}; brainstorm uses for \emph{a hard drive}; and so on. The participant chooses one sub-problem at a time and generates solutions, each of which is also a solution to the original problem. For example, the following are generated by a participant focussing on the audio output capabilities of an MP3 player:

\begin{itemize}
    \item have them as a resource on public transportation -- people must supply their own headphones
    \item use them in museums to give information on various installations
    \item have cities install them in tourist areas, so people can listen about where they are
\end{itemize}

The previous examples of problem scoping are examples of emphasizing specific details of the problem. Problem scoping can also entail neglecting details. For example, the following ideas from a participant apply to almost any physical object, not just the MP3 player described:

\begin{itemize}
    \item doorstop
    \item use in abstract art
    \item recycle
\end{itemize}

We refer to the former, detail-oriented kind of problem scoping as \emph{scoping inward} and the alternative as \emph{scoping outward}. Problem scoping is employed by all participants, but the type of scoping (inwards vs outward) is a useful discriminant.

\subsubsection{Riffing}

The second category of strategies is \emph{riffing}.
We found riffing manifested itself many ways, and identified four distinct riffing strategies used by participants: specificity riffing, exhaustive mutation, exploratory riffing, and follow-up riffing. We describe each of these in detail, and then touch on three common phenomena found under riffing strategies.

\emph{Generalization riffing} occurs when a participant generates two or more ideas and one is a generalization of the other. For example, two consecutive ideas given by a participant:

\begin{itemize}
    \item brick
    \item building material
\end{itemize}

Another type of riffing is \emph{exhaustive mutation}. Under this strategy, a participant repeats a response nearly verbatim many times, each time replacing a single word or concept in such a way that no new ideas are formed. For example:

\begin{itemize}
    \item wrap yarn around it
    \item wrap twine around it
    \item wrap string around it
\end{itemize}

\emph{Hold riffing} is a strategy in which participants hold at least one element of a previous response constant when generating a new response, such that a new idea is formed. This can be thought of as analogous to the game of \emph{draw poker}, in which players hold some cards while receiving new ones, with the intent of building a winning hand complementing the held cards.
For example:

\begin{itemize}
    \item Jukebox music selector in bars
    \item Commercials in bar bathrooms (hold the bar element)
    \item Tapper handles for beer (hold the alcohol element)
\end{itemize}

Finally, \emph{Continuation riffing} is the strategy of creating another idea that assumes the previous idea has already been implemented. The new ideas relies on the implications or consequences of the previous. For example:

\begin{itemize}
    \item I suppose you could just grind them down into a sand
    \item You could take the sand... and put it in an hourglass
\end{itemize}

Within and across these strategies, there are a few repeated phenomena.
Some participants \emph{pair} their riffs. These participants  often riff on old ideas but never produce more than two versions of each idea. This may represent a desire to generate more ideas to meet the task expectations, but a trepitation that repeated ideas are less satisfying than completely unique ones.

Another phenomenon is \emph{linguistic riffing}. Some participants seem to use common sentence structure as queues when generating ideas. For example:

\begin{itemize}
    \item We could use the hard-drives inside for different electronics.
    \item We could use them in place of rocks (to throw at things, to use in pavement.)
    \item We could use them to back up our music collections on our current devices.
\end{itemize}

Note the distinction from exhaustive mutation. In the exhaustive mutation example above, instances using the same language were interchangeable.
In this example, while language is re-used, each instance is a new idea.

Finally, \emph{reaching back} is a variant of riffing employed by almost all participants who provided more than 10 responses. Reaching back entails riffing on an idea that occured further back in the run than the previous response. The presence of reaching back speaks to the exhaustion of the participant's creativity, as they produce new variants of old ideas rather than new ideas altogether.

Almost all participants employ some form of riffing. This is not surprising, as it is ingrained in the rules of brainstorming given at the beginning of the task: "combinations of ideas count as new ideas". However, the different strategies of riffing have very different impacts on the qualitative assessment of a run. Exhaustive mutation is often unsatisfying, as it feels as though the participant is trying to wring every scrap of task completion out of a single idea. On the other hand, exploratory riffing can produce answers that are unexpected and of high quality. 

\subsubsection{Time-saving}

Two time-saving strategies were also identified. The first, perhaps overly obvious strategy, is \emph{brevity}. This is simply the act of keeping responses as concise as possible, presumably in the interest of finishing quickly. The second time-saving strategy was \emph{cop-outs}. Cop-outs occur when participants choose to resort to vague or iunrelated responses that do not actually provide any solution to the problem. For example, one participant's response to the MP3 player problem was to "Remove glass screen to make something". Cop-out responses may provide one or more steps towards a solution, but no solution.

Time-saving strategies tend to negatively impact the quality of responses. While there are succinct, creative responses, many brief responses are often a sign that the ideas within the run are "low-hanging fruit". Cop-out ideas are poor quality by definition.

\subsubsection{Humour}

Some participants used \emph{humour} as a method of generating ideas. For example, one response given was "glue it to the side walk and let the fun ensue". While ideas generated under this strategy are impractical individually, this strategy is most often employed by participants who otherwise take the idea generation task seriously and perform successfully.

\subsection{Personas}

We use personas to capture collections of strategy use we see repeated across participants. 
Six personas were identified: quitters, lazy creatives, diligent struggles, lazy strugglers, hard workers, and over-deliverers. These personas are described below. A summary, in terms of strategies used, is in Table TAB.

\emph{Quitters} are the easiest to summarize. These participants do not provide all the requested responses, and often resort to cop-outs or exhaustive mutation in the ideas that they do provide.

\emph{Lazy creatives} provide primarily uncommon responses, but do not put effort into the responses. They employ brevity, reach back and exhaustive mutation strategies to a great extent. These participants seem like prime targets for future interventions, as their ideas are of interest but they focus their strategies towards rapid completion. Lazy creatives employ minimal feature utilization.

\emph{Diligent strugglers} are the opposite of lazy creatives. These participants do not employ the brevity or exhaustive mutation strategies, but nonetheless produce primarily common, unoriginal ideas. These participants also seem like potential targets for interventions, as they apply themselves to the task but may need external assistance to move beyond their comfort zone. Strategies employed include exploratory riffing and feature utilization.

\emph{Lazy strugglers} are simply those who produce no creative ideas and are predisposed towards unproductive strategies (primarily exhaustive mutation and brevity).

\emph{Hard workers} are notable for their minimal use of riffing strategies. The riffing strategies they do use tend to exclude exhaustive mutation. While brevity is occasionally employed for short periods, it is generally only where a short description is sufficient to communicate an idea.

\emph{Over-deliverers} are notable for their lack of strategy in general. Their ideas are verbose, distinct and feature extremely minimal riffing. 

\begin{table*}
    \begin{tabular}{|l|p{0.35\linewidth}|l|}
        \hline
        \textbf{Persona} & \textbf{Common strategies} & \textbf{Rare strategies} \\
        \hline
        Quitters & cop-outs, exhaustive mutation & follow-up riffing \\
        \hline
        Lazy creatives & exploratory riffing, specificity riffing, \newline
        exhaustive mutation, brevity, reach-backs  & follow-up riffing  \\
        \hline
        Diligent strugglers & exploratory riffing & exhaustive mutation, brevity, \newline
        follow-up riffing \\
        \hline
        Lazy strugglers & exhaustive mutation, exploratory riffing, \newline
        cop-outs, brevity  & follow-up riffing \\
        \hline
        Hard workers & feature use, humour, \newline
        exploratory riffing, specificity riffing  & follow-up riffing, exhaustive mutation \\
        \hline
        Over-deliverers & feature use & riffing, reach-backs, brevity \\
        \hline
    \end{tabular}
    \caption{Strategies used by personas}
\end{table*}
