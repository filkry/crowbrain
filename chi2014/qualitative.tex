\section{Qualitative findings}

In addition to modelling data and confirming our hypotheses, we performed a qualitative analysis.
We selected 30 brainstorming runs (5 from each condition). The primary author applied open coding to identify characteristics of responses and runs that are independent of the uniqueness and originality measures observed above. In particular, we identified groups of ideas within runs that shared characteristics. These characteristics, in addition to temporal order, form the basis for what we call \emph{strategies}: repeatedly observed classes of ideas that are generated by some participants but not others.
Furthermore, we use the presence or lack of presence of strategies to define \emph{personas}.

\subsection{Strategies}

A \emph{strategy} is a class of ideas repeatedly observed in groups, though not neccessarily consecutively, within brainstorming runs. These ideas are defined by some common characteristics and may or may not be further defined by temporal order. Not all participants utilized all strategies in their brainstorming runs.

We identified four categories of strategy. The first, \emph{feature exploration}, is specific to the MP3 player problem presented in the paper. This category entails focussing on specific features a device possesses and leveraging them as the basis for generating new ideas. The second category expands on the \emph{riffing} phenomenon introduced earlier, identifying six(?) different types of riffing exployed. \emph{Time-saving} strategies are those that allow the participant to generate at a faster rate. Finally, the \emph{humour} strategy emphasizes generating ideas that are amusing rather than practical.

\subsubsection{Feature exploration}

One category of strategies employed was \emph{feature exploration}. An MP3 player has many features that distinguish it from any other electronic device of the same physical proportions: audio output (and potentially input), data storage, a screen, general purpose computing abilities (apps), connectivity (WiFi, USB), a camera, a unique appearance, GPS.
Each of these features forms the basis for a strategy of idea generation.
Participants would choose a feature and generate ideas that leveraged that feature in various ways. For example, here are several answers from a participant who explored audio input and output:

\begin{itemize}
    \item have them as a resource on public transportation -- people must supply their own headphones
    \item use them in museums to give information on various installations
    \item have cities install them in tourist areas, so people can listen about where they are
\end{itemize}

While it might be expected that all participants would employ this strategy to some degree, many treated the MP3 player as a generic electronic device of equivalent size. Runs were commonly composed of answers that could apply to almost any un-used household item:

\begin{itemize}
    \item doorstop
    \item use in abstract art
    \item recycle
\end{itemize}

Runs that used a feature exploration strategy produce responses that seem intuitively better than those that do not. It is satisfying to have a use for a device that "fits" the device as closely as possible, so that none of the functionality is "wasted". 

\subsubsection{Riffing}

The second category of strategies is \emph{riffing}.
We found riffing manifested itself many ways, and identified four distinct riffing strategies used by participants: specificity riffing, exhaustive mutation, exploratory riffing, and follow-up riffing. We describe each of these in detail, and then touch on three common phenomena found under riffing strategies.

\emph{Specificity riffing} occurs when a participant generates two or more ideas and one is a generalization of the other. For example, two consecutive ideas given by a participant:

\begin{itemize}
    \item brick
    \item building material
\end{itemize}

Another, more common type of riffing is \emph{exhaustive mutation}. Under this strategy, a participant repeats a response nearly verbatim many times, each time replacing a single word or concept in such a way that no new ideas are formed. For example:

\begin{itemize}
    \item wrap yarn around it
    \item wrap twine around it
    \item wrap string around it
\end{itemize}

\emph{Exploratory riffing} is a strategy in which participants hold at least one element of a previous response constant when generating a new response, such that a new idea is formed. This can be chained many times in a kind of idea "telephone game" such that the final ideas no longer resemble the original source. For example:

\begin{itemize}
    \item Jukebox music selector in bars
    \item Commercials in bar bathrooms (retains bar element)
    \item Tapper handles for beer (retain alcohol element)
\end{itemize}

Finally, \emph{Follow-up riffing} occurs when a participant's response assumes that the solution in one of their previous responses has already been implemented. For example:

\begin{itemize}
    \item I suppose you could just grind them down into a sand
    \item You could take the sand... and put it in an hourglass
\end{itemize}

Within and across these strategies, there are a few repeated phenomena.
Some participants \emph{pair} their riffs. These participants  often riff on old ideas but never produce more than two versions of each idea. This may represent a desire to generate more ideas to meet the task expectations, but a trepitation that repeated ideas are less satisfying than completely unique ones.

Another phenomenon is \emph{linguistic riffing}. Some participants seem to use common sentence structure as queues when generating ideas. For example:

\begin{itemize}
    \item We could use the hard-drives inside for different electronics.
    \item We could use them in place of rocks (to throw at things, to use in pavement.)
    \item We could use them to back up our music collections on our current devices.
\end{itemize}

Note the distinction from exhaustive mutation. In the exhaustive mutation example above, instances using the same language were interchangeable.
In this example, while language is re-used, each instance is a new idea.

Finally, \emph{reaching back} is a variant of riffing employed by almost all participants who provided more than 10 responses. Reaching back entails riffing on an idea that occured further back in the run than the previous response. The presence of reaching back speaks to the exhaustion of the participant's creativity, as they produce new variants of old ideas rather than new ideas altogether.

Almost all participants employ some form of riffing. This is not surprising, as it is ingrained in the rules of brainstorming given at the beginning of the task: "combinations of ideas count as new ideas". However, the different strategies of riffing have very different impacts on the qualitative assessment of a run. Exhaustive mutation is often unsatisfying, as it feels as though the participant is trying to wring every scrap of task completion out of a single idea. On the other hand, exploratory riffing can produce answers that are unexpected and of high quality. 

\subsubsection{Time-saving}

Two time-saving strategies were also identified. The first, perhaps overly obvious strategy, is \emph{brevity}. This is simply the act of keeping responses as concise as possible, presumably in the interest of finishing quickly. The second time-saving strategy was \emph{cop-outs}. Cop-outs occur when participants choose to resort to vague or iunrelated responses that do not actually provide any solution to the problem. For example, one participant's response to the MP3 player problem was to "Remove glass screen to make something". Cop-out responses may provide one or more steps towards a solution, but no solution.

Time-saving strategies tend to negatively impact the quality of responses. While there are succinct, creative responses, many brief responses are often a sign that the ideas within the run are "low-hanging fruit". Cop-out ideas are poor quality by definition.

\subsubsection{Humour}

Some participants used \emph{humour} as a method of generating ideas. For example, one response given was "glue it to the side walk and let the fun ensue". While ideas generated under this strategy are impractical individually, this strategy is most often employed by participants who otherwise take the idea generation task seriously and perform successfully.

\subsection{Personas}

We use personas to capture collections of strategy use we see repeated across participants. 
Six personas were identified: quitters, lazy creatives, diligent struggles, lazy strugglers, hard workers, and over-deliverers. These personas are described below. A summary, in terms of strategies used, is in Table TAB.

\emph{Quitters} are the easiest to summarize. These participants do not provide all the requested responses, and often resort to cop-outs or exhaustive mutation in the ideas that they do provide.

\emph{Lazy creatives} provide primarily uncommon responses, but do not put effort into the responses. They employ brevity, reach back and exhaustive mutation strategies to a great extent. These participants seem like prime targets for future interventions, as their ideas are of interest but they focus their strategies towards rapid completion. Lazy creatives employ minimal feature utilization.

\emph{Diligent strugglers} are the opposite of lazy creatives. These participants do not employ the brevity or exhaustive mutation strategies, but nonetheless produce primarily common, unoriginal ideas. These participants also seem like potential targets for interventions, as they apply themselves to the task but may need external assistance to move beyond their comfort zone. Strategies employed include exploratory riffing and feature utilization.

\emph{Lazy strugglers} are simply those who produce no creative ideas and are predisposed towards unproductive strategies (primarily exhaustive mutation and brevity).

\emph{Hard workers} are notable for their minimal use of riffing strategies. The riffing strategies they do use tend to exclude exhaustive mutation. While brevity is occasionally employed for short periods, it is generally only where a short description is sufficient to communicate an idea.

\emph{Over-deliverers} are notable for their lack of strategy in general. Their ideas are verbose, distinct and feature extremely minimal riffing. 

\begin{table*}
    \begin{tabular}{|l|p{0.35\linewidth}|l|}
        \hline
        \textbf{Persona} & \textbf{Common strategies} & \textbf{Rare strategies} \\
        \hline
        Quitters & cop-outs, exhaustive mutation & follow-up riffing \\
        \hline
        Lazy creatives & exploratory riffing, specificity riffing, \newline
        exhaustive mutation, brevity, reach-backs  & follow-up riffing  \\
        \hline
        Diligent strugglers & exploratory riffing & exhaustive mutation, brevity, \newline
        follow-up riffing \\
        \hline
        Lazy strugglers & exhaustive mutation, exploratory riffing, \newline
        cop-outs, brevity  & follow-up riffing \\
        \hline
        Hard workers & feature use, humour, \newline
        exploratory riffing, specificity riffing  & follow-up riffing, exhaustive mutation \\
        \hline
        Over-deliverers & feature use & riffing, reach-backs, brevity \\
        \hline
    \end{tabular}
    \caption{Strategies used by personas}
\end{table*}
