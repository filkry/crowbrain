\section{Conclusion}

We set out to establish a baseline for brainstorming performance in crowd marketplaces, a new setting for idea generation in with spatial and temporal co-presence are not guaranteed. We did this by establishing a corpus of brainstorming responses, publically available at URL. Furthermore, we introduce the \emph{idea forest}, an encoding of brainstorming responses that captures differences in generality we discovered, an introduced an algorithm for producing idea forests.

We found that originality increased as brainstorming participants were asked to generate more ideas individually. Generality, however, decreased. In both cases, we found a ``sweet spot" for this change at the 20 idea mark; participants' ideas reach their peak creativity and minimum generality after their 20th contribution.

Furthermore, we verified that the findings and predictions of prior work in which participants were spatially and temporally copresent. We found that, in line with the predictions of Jigstad and Stroebe's SIAM model \cite{nijstad_how_2006}, ideas are more likely to follow an idea of the same category than would be randomly expected, and that the time spent to generate an idea within-category is lesser than that to generate ideas between categories. Furthermore, we recreated the finding of Parnes et al \cite{parnes_effects_1961} that responses in the latter part of a participant's idea generation process are more creative than those earlier.

We identified three qualitative phenomena in of brainstorming results. First, that brainstorming results are subject to the popular consciousness of the time they are gathered. Second, that workers are more likely to take HITs with higher absolute payout, even if relative payout is identical.

The most salient recommendation we can make is easily summarized: if you use the crowd for brainstorming and desire originality, it is most productive to ask few participants for more than 20 ideas each.