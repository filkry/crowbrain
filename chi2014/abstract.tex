Microtask marketplaces enable hybrid computational systems that use crowds to solve problems of creativity, such as brainstorming. We present a first characterization of brainstorming in this context, providing models for the rate of idea generation, the rate of category generation and the originality of generated ideas. We demonstrate that the rates of idea and category generation can be closely fit by a logarithmic model, and show that the first few responses of a brainstorming run are drawn from a pool of very common, obvious ideas. Our results suggest a cutoff for this strategy 20th idea. Furthermore, we explore the phenomenon of \emph{riffing} in brainstorming tasks, and replicate two related findings from prior work. Our results provide recommendations for those leveraging the crowd for brainstorming, and comparable models of creative idea generation research community.


%Microtask marketplaces enable hybrid computational systems that use crowds to solve problems of creativity, such as brainstorming. We present a first characterization of brainstorming in this context, providing models for the rate of idea generation, the rate of category generation, the originality of ideas and the novelty of ideas. We demonstrate that the rates of idea and category generation can be closely fit by a logarithmic model, that these rates roughly increase if fewer participants are asked for more responses, and that this distinction is the result of a \emph{burn in period}  of common ideas at the beginning of a brainstorming task. Furthermore, we explore the phenomenon of \emph{riffing} in brainstorming tasks, and replicate two related findings from prior work. Our results provide recommendations for those leveraging the crowd for brainstorming, and comparable models of creative idea generation research community.
