\section{Future Work}

%The most immediate step that follows this work is to examine the same phenomena for multiple questions. A design space for brainstorming taks must be defined and tested to see how variations impact the rate of idea generation, originality, etc.


%The saturation point of unique ideas and categories presents an obstacle to large-scale brainstorming on microtask marketplaces. One could conceivably develop interventions to overcome and reset this stagnation behaviour. For example, ideas from previous responders could be interspersed in response fields of the task to promote riffing or combination. 

The next step following this work is to explore interventions to overcome the saturation point of unique ideas and categories presents an obstacle to large-scale brainstorming on microtask marketplaces. For example, ideas from previous responders could be strategically interspersed in response fields of the task to promote riffing or exploration of a design space in certain directions. 

Our results show that ideas are more likely to be followed by one from the same category, and that changing categories takes time. Combining these models with others (such as natural language measures for text similarity), one could construct a posterior belief model to guess whether a participant is currently riffing. 

Once the point of riffing and the current category a participant is working on can be identified, we can come up with different mechanisms for crowd to explore a design space. One could imagine a brainstorming session where participants are encouraged to generate a lot of ideas within new or rare categories or discouraged from exhausting categories that are popular.



%The change in idea and category o-scores at the 20 instance point is reflected in other metrics not discussed in this paper. It is likely this represents a distinct strategy change. By modelling this strategy change explicitly, we may be able to identify more explicitly when this change happens and make predictions about later strategies based on those employed early in the brainstorming run.


