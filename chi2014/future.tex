\section{Future Work}

The most immediate work that follows from this is to examine the same phenomena for multiple questions. A design space for brainstorming taks must be defined and tested to see how variations impact the rate of idea generation, originality, etc.

The saturation point of unique ideas and categories presents an obstacle to large-scale brainstorming on microtask marketplaces. One could conceivable develop interventions to overcome and reset this stagnation behaviour. For example, ideas from previous responders could be interspersed in response fields of the task to promote riffing or combination.

We identified that ideas are more likely to be followed by one from the same category, and that changing categories takes time. Combining these models with others (such a natural language measures for text similarity), one could construct a posterior belief model for whether a participant is currently riffing.

The change in idea and category o-scores at the 20 instance point is reflected in other metrics not discussed in this paper. It is likely this represents a distinct strategy change. By modelling this strategy change explicitly, we may be able to identify more explicitly when this change happens and make predictions about later strategies based on those employed early in the brainstorming run.


